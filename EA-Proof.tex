\documentclass[11pt]{article}
\usepackage[a4paper,margin=1in]{geometry}
\usepackage{amsmath,amssymb,amsthm,stmaryrd,mathpartir}
\usepackage{hyperref}

\title{Appendix A\\A Gentzen-Style Sequent Proof of the Existence Axiom}
\author{(to accompany “The Existence Axiom – A Minimal-Commitment Metaphysics for Science”)}
\date{}

% ------ Custom commands ------
\newcommand{\bx}{\Box}
\newcommand{\dia}{\Diamond}
\newcommand{\Px}{P(x)}
\newcommand{\G}{\mathcal{G}}
\newcommand{\seq}{\Rightarrow}

% ------ The proof environment ------
\theoremstyle{definition}
\newtheorem{lemma}{Lemma}[section]
\newtheorem{theorem}{Theorem}[section]

\begin{document}
\maketitle
\tableofcontents
\newpage

\section{Preliminaries}

\subsection{Language $L$}
First-order modal language with identity.  
Predicate $P$ denotes “is an instantiated process.”  

\subsection{Semantics}
Classical S5 Kripke frames $\langle W,R\rangle$ with $R$ an equivalence relation.  
All ordinary worlds $w\in W$ have non-empty domains $D(w)\neq\varnothing$.  

\subsection{Meta-semantic Condition}
Any well-formed description of a model employs a metalanguage $M$ containing
\begin{itemize}
\item at least one concrete symbol token $\sigma_i$,
\item at least one inference rule $\rho_j$,
\end{itemize}
hence constitutes an \emph{instantiated process}.  
This fact will underwrite the Reference Lemma below.

\section{Target Theorem}

\begin{theorem}[Existence Axiom]\label{thm:EA}
\[
\bx\exists x\,P(x)
\]
Necessarily, some instantiated process exists.
\end{theorem}

\section{Key Lemmas}

\begin{lemma}[Reference Lemma]\label{lem:ref}
Every act of model specification is itself an instantiated process; formally,
\[ \text{``$\mathfrak{M}$ is a model of $L$''}\;\rightarrow\; \exists x\,P(x).\]
\end{lemma}
\begin{proof}
Whatever physically realises the token string “$\mathfrak{M}$” plus the rules that licence its use is a concrete dynamical structure; by definition it satisfies $P$.
\end{proof}

\begin{lemma}[Empty-Domain Incoherence]\label{lem:empty}
A world with $D(w)=\varnothing$ cannot be uniquely denoted without violating Lemma \ref{lem:ref}.
\end{lemma}
\begin{proof}
Denoting $w$ requires the antecedent act of reference, hence a process.  
If $D(w)=\varnothing$, then $w$ lacks even that process, contradicting the Reference Lemma.
\end{proof}

\begin{lemma}[Modal Transfer]\label{lem:transfer}
If $\lnot\varphi$ is self-defeating in $M$, then $\bx\varphi$.
\end{lemma}
\begin{proof}
Assume $\dia\lnot\varphi$; in some accessible $w$ we have $\lnot\varphi$.  
If $\lnot\varphi$’s very statement entails a process excluded by $w$, we face contradiction; hence $\dia\lnot\varphi$ is impossible and $\bx\varphi$ follows by S5 duality.
\end{proof}

\section{Gentzen-Style Sequent Derivation}

\subsection*{Abbreviations}
$\G$ ranges over finite sets of formulas.  
Standard sequent $\G\seq\Delta$ rules for first-order S5 are assumed.

\bigskip
\noindent\textbf{Goal sequent:}\quad
$\seq\bx\exists x\,P(x)$

\begin{mathpar}
\inferrule*[Right=$\bx$-R]{%
  \seq \exists x\,P(x)
}{%
  \seq \bx\exists x\,P(x)
}
\end{mathpar}

Thus it suffices to derive the \emph{subgoal} $\seq\exists x\,P(x)$.  
Proceed by \emph{reductio}:

\bigskip
\noindent\textbf{Assumption} (A): $\lnot\exists x\,P(x)$.

\begin{mathpar}

\inferrule*[Right=$\lnot$-L]{%
  \exists x\,P(x) \seq
}{%
  \lnot\exists x\,P(x),\;\exists x\,P(x) \seq
}

\end{mathpar}

Left branch closes by identity; right branch must show $\seq\,$(empty succedent) is untenable because Lemma \ref{lem:ref} forces $\exists x\,P(x)$.

\bigskip
\noindent\textbf{Metalogical step (Reference Lemma):}

\begin{mathpar}
\inferrule*[Right=Ref]{%
}{%
  \seq \exists x\,P(x)
}
\end{mathpar}

This sequent is an initial axiom under Gentzen’s \emph{external} reasoning: the proof’s own existence instantiates $P$.

\bigskip
Combining with assumption (A) yields an immediate contradiction, so $\lnot\exists x\,P(x)$ cannot appear undischarged; hence $\seq\exists x\,P(x)$ is proved, and by $\bx$-necessitation we obtain Theorem \ref{thm:EA}.

\section{Line-by-Line S5 Table (Expanded)}

\begin{enumerate}
\item $\Px \seq \Px$ \hfill ID
\item $\seq \forall x\,P(x)\rightarrow P(a)$ \hfill $\forall$-R
\item $\seq \lnot\exists x\,P(x) \rightarrow \lnot P(a)$ \hfill quantifiers
\item $\seq \bx\lnot\exists x\,P(x) \rightarrow \bx\lnot P(a)$ \hfill K
\item $\seq \dia P(a) \rightarrow \dia\exists x\,P(x)$ \hfill contraposition
\item $\seq \lnot\dia\lnot\exists x\,P(x) \rightarrow \bx\exists x\,P(x)$ \hfill axiom 5
\item $\seq \bx\exists x\,P(x)$ \hfill nec.\,+\;MP using steps 1–6
\end{enumerate}

\section{Discussion of the Metalogical Pivot}

Step 1 of the Gentzen reduction imported Lemma \ref{lem:ref} as an \emph{extra-systemic} justification—orthodox in dialetheic treatments of self-reference but unusual in everyday modal proofs. The manoeuvre is legitimate because EA’s target is precisely the impossibility of excluding\,\slash\,quantifying over the very reference required to state that exclusion.

\section{Relation to Empty-Domain Semantics}

Some logicians allow empty domains by tinkering with quantifier clauses.  
EA’s proof blocks \emph{absolute} emptiness by showing that any metalanguage enabling such tinkering already commits us to processes, moving emptiness to a merely \emph{relative} notion.

\section{Concluding Remark}

The derivation secures $\bx\exists x\,P(x)$ with nothing stronger than classical S5 plus the unavoidable reference facts about formal discourse. No contentious metaphysical premise beyond those facts is invoked.

\end{document}

\documentclass[12pt,a4paper]{article}
\usepackage[utf8]{inputenc}
\usepackage{amsmath, amssymb, amsthm}
\usepackage{geometry}
\usepackage{hyperref}
\usepackage{enumitem}

\geometry{margin=1in}

\title{A Formal Inconsistency Proof of Absolute Nothingness via Kolmogorov Complexity and the Minimal Information Instantiation Principle}
\author{Liam Love Sköld\\\textit{Independent}}
\date{}

% Theorem environment
\newtheorem{theorem}{Theorem}

\begin{document}

\maketitle

\begin{abstract}
This paper offers a formal argument showing that \emph{absolute nothingness}---the total absence of entities, properties, and information---is logically inconsistent within the framework of Zermelo-Fraenkel set theory (ZFC), Kolmogorov complexity, and the Minimal Information Instantiation Principle (MIIP), introduced here. The claim is that any instantiable object or state in such a system must possess at least irreducible descriptive content. Since ``absolute nothingness'' lacks even this minimal information, its instantiation is contradictory. This result has philosophical implications, suggesting ``nothing'' is not just absent in reality but incoherent as a formal concept. The paper concludes with reflections for metaphysics, modal logic, and information theory.
\end{abstract}

\section{Introduction}

The concept of ``absolute nothingness''---a state lacking objects, properties, relations, and information---has long challenged both philosophy and physics. Classical positions, such as Parmenides’ denial of non-being and Heidegger’s inquiry into why there is something rather than nothing, reveal the depth of this issue. Quine’s discussion of ontological commitment further illuminates how language and logic shape notions of existence.

Mathematically, Zermelo-Fraenkel set theory models ``nothing'' with the empty set, but even this is a definable entity: a ``something'' within the framework. What about the stronger notion of absolute nothingness---the total absence of all entities and information?

By leveraging algorithmic information theory and introducing the Minimal Information Instantiation Principle (MIIP)---a generalization of Leibniz’s Principle of Sufficient Reason---this paper contends that ``absolute nothingness'' is not just absent in reality, but formally inconsistent.

\section{Formal Preliminaries and Key Definitions}

\subsection{Zermelo-Fraenkel Set Theory (ZFC)}

Modern mathematics uses ZFC as a foundational system for discussing objects and their properties \cite{Jech2003}.

\subsection{Kolmogorov Complexity}

Kolmogorov complexity measures the shortest program capable of generating a specific output; differences in results between universal machines are bounded by a constant and do not affect the argument’s thrust \cite{Kolmogorov1965, Chaitin1966, LiVitanyi2008}.

\subsection{Minimal Information Instantiation Principle (MIIP)}

MIIP, introduced here, stipulates:

\begin{quote}
Any entity or state that can be instantiated within a formal system must be associated with at least a small, nonzero quantity of descriptive information.
\end{quote}

Formally, for any possible entity $e$, there exists a description $d_e$ such that
\[
K(d_e) \geq c > 0,
\]
where $c$ is a lower bound inherent to the system.

\subsection{Absolute Nothingness ($\mathcal{N}$)}

Here, ``absolute nothingness'' refers to a state with no entities, properties, relations, or instantiable structure---realized as a complete absence, even of information.

\section{Theorem and Proof}

\begin{theorem}
There is no instantiable state $\mathcal{N}$ representing absolute nothingness within the combined framework of ZFC, Kolmogorov complexity, and MIIP.
\end{theorem}

\begin{proof}
Suppose, for contradiction, that such a state $\mathcal{N}$ can be instantiated:

\begin{enumerate}[label=(\alph*)]
  \item By assumption, $\mathcal{N}$ is formally definable within the system.
  \item By definition, $\mathcal{N}$ has no information and its description’s complexity satisfies
  \[
  K(D(\mathcal{N})) = 0.
  \]
  \item According to MIIP,
  \[
  K(D(\mathcal{N})) \geq c > 0
  \]
  for any instantiated entity.
\end{enumerate}

Therefore,
\[
K(D(\mathcal{N})) = 0 \quad \wedge \quad K(D(\mathcal{N})) \geq c > 0,
\]
which is a contradiction. It follows that absolute nothingness cannot be instantiated.
\end{proof}

\paragraph{On the Empty Set}

It is worth clarifying that the empty set $\varnothing$ is often thought to represent ``nothing.'' Yet, as a mathematically definable object with nonzero descriptive complexity, it does not correspond to true absolute nothingness, but rather to an empty mathematical entity \cite{Jech2003}.

\section{Discussion}

\subsection{MIIP and Sufficient Reason}

MIIP expresses the idea that existence requires information. This connects to Leibniz’s Principle of Sufficient Reason \cite{Leibniz1714} and to recent ideas in digital physics, in which the fabric of reality is viewed as fundamentally informational \cite{Tegmark2008}.

\subsection{Beyond Epistemology}

This line of argument is not about what we can (or cannot) know about ``nothing,'' but rather claims that the very instantiation of absolute nothingness is logically inconsistent.

\subsection{Modal and Ontological Implications}

In modal terms,
\[
\neg \Diamond \mathcal{N},
\]
expresses the impossibility of instantiating absolute nothingness, supporting the intuition that ``there is always something.''

\section{Conclusion}

By combining set theory, Kolmogorov complexity, and a new metaphysical principle, it emerges that ``absolute nothingness'' is not just physically absent—it is formally inconsistent. Any instantiable system inevitably contains some information; thus, existence appears to be fundamentally informational.

\begin{thebibliography}{99}

\bibitem{Chaitin1966}
Chaitin, G. J. (1966). On the length of programs for computing finite binary sequences. \emph{Journal of the ACM}, 13(4), 547--569.

\bibitem{Heidegger1962}
Heidegger, M. (1962). \emph{Being and Time} (J. Macquarrie \& E. Robinson, Trans.). Harper \& Row. (Original work published 1927)

\bibitem{Jech2003}
Jech, T. (2003). \emph{Set Theory}. Springer.

\bibitem{Kolmogorov1965}
Kolmogorov, A. N. (1965). Three approaches to the quantitative definition of information. \emph{Problems of Information Transmission}, 1(1), 1--7.

\bibitem{Leibniz1714}
Leibniz, G. W. (1714). \emph{Monadology}.

\bibitem{LiVitanyi2008}
Li, M., \& Vitányi, P. (2008). \emph{An Introduction to Kolmogorov Complexity and Its Applications} (3rd ed.). Springer.

\bibitem{Quine1948}
Quine, W. V. (1948). On what there is. \emph{Review of Metaphysics}, 2(5), 21--38.

\bibitem{Tegmark2008}
Tegmark, M. (2008). The mathematical universe. \emph{Foundations of Physics}, 38(2), 101--150.

\end{thebibliography}

\end{document}

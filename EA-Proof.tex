\documentclass[12pt,a4paper]{article}

\usepackage[utf8]{inputenc}
\usepackage{amsmath, amssymb, amsthm}
\usepackage{geometry}
\usepackage{hyperref}
\usepackage{enumitem}
\usepackage{mathrsfs}

\geometry{margin=1in}

\title{A Formal Inconsistency Proof of Absolute Nothingness via Kolmogorov Complexity and the Minimal Information Instantiation Principle}
\author{Liam Love Sköld\\\textit{Independent}}
\date{}

% Theorem environment
\newtheorem{theorem}{Theorem}

\begin{document}

\maketitle

\begin{abstract}
This paper presents a formal proof demonstrating that absolute nothingness---the total absence of entities, properties, and even information---is logically inconsistent within a system combining Zermelo-Fraenkel set theory (ZFC), Kolmogorov complexity, and the Minimal Information Instantiation Principle (MIIP), an original principle introduced herein. We argue that any instantiable object or state within such a system necessarily requires minimal irreducible descriptive content. Since absolute nothingness entails the absence of even this minimal information, its instantiation leads to contradiction. This conclusion engages with longstanding metaphysical debates, suggesting that ``nothing'' is not merely absent in reality but is incoherent as a formal object. We conclude with implications for metaphysics, modal logic, and information theory.
\end{abstract}

\section{Introduction}

The notion of ``absolute nothingness''---a state devoid of objects, properties, relations, or any informational content---has perplexed philosophers and physicists alike. Classical philosophical investigations, such as Parmenides’ denial of non-being and Heidegger’s inquiry into why there is something rather than nothing \cite{Heidegger1962}, underscore the depth of this challenge. Similarly, Quine’s work on ontological commitment \cite{Quine1948} clarifies how language and logic shape what can be said to ``exist.''

In formal systems like Zermelo-Fraenkel set theory (ZFC) \cite{Jech2003}, ``nothing'' is often modeled by the empty set, yet this substitution obscures a deeper issue: even the empty set is something---a definable entity within a mathematical framework. What, then, of absolute nothingness---a total absence of all structure and instantiation?

Drawing on algorithmic information theory \cite{Kolmogorov1965, Chaitin1966, LiVitanyi2008}, this paper introduces the Minimal Information Instantiation Principle (MIIP), which generalizes Leibniz’s Principle of Sufficient Reason \cite{Leibniz1714}. MIIP posits that any instantiable entity must contain at least a finite, nonzero quantity of descriptive information. Within this framework, the very notion of absolute nothingness becomes inconsistent.

\section{Formal Preliminaries and Key Definitions}

\subsection{Zermelo-Fraenkel Set Theory (ZFC)}

ZFC provides the foundational axiomatic framework of modern set theory \cite{Jech2003}, forming the backdrop for formalizing mathematical objects and their properties.

\subsection{Kolmogorov Complexity ($K$)}

Kolmogorov complexity quantifies the minimal length of a program that outputs a given string on a universal Turing machine \cite{Kolmogorov1965, Chaitin1966}. Despite dependence on the choice of universal machine, differences are bounded by a constant and thus do not affect our formal conclusions \cite{LiVitanyi2008}.

\subsection{Minimal Information Instantiation Principle (MIIP)}

MIIP is an original metaphysical-formal principle introduced here:

\begin{quote}
Any entity or state instantiated within a formal system must be associated with a finite, nonzero amount of descriptive information.
\end{quote}

Formally, for any instantiable entity $e_e$, there exists a description $d_e$ such that
\[
K(d_e) \geq c > 0,
\]
where $c$ is a system-invariant lower bound.

This principle builds upon Leibniz’s classical insight \cite{Leibniz1714} and contemporary ideas in algorithmic metaphysics and digital ontology \cite{Tegmark2008}.

\subsection{Absolute Nothingness ($\mathcal{N}$)}

We define absolute nothingness as the ontological state containing no entities, properties, relations, information, or instantiable structure---stronger than the empty set or vacuum in physics.

\section{Theorem and Proof}

\begin{theorem}
There is no instantiable state $\mathcal{N}$ representing absolute nothingness within the system combining ZFC, MIIP, and Kolmogorov complexity.
\end{theorem}

\begin{proof}
Assume, for contradiction, that such a state $\mathcal{N}$ exists:
\[
\exists \mathcal{N} \text{ such that } \mathcal{N} \text{ is definable in ZFC + MIIP.}
\]

By definition,
\[
\neg \exists x \ (x = x), \quad \text{and} \quad K(D(\mathcal{N})) = 0,
\]
since $\mathcal{N}$ contains no information.

By MIIP, any instantiated entity must satisfy
\[
K(D(\mathcal{N})) \geq c > 0.
\]

This contradicts the previous point, implying
\[
K(D(\mathcal{N})) = 0 \quad\wedge\quad K(D(\mathcal{N})) \geq c > 0.
\]

Therefore, $\mathcal{N}$ cannot be instantiated.
\end{proof}

\paragraph{Clarifying Note on the Empty Set}

The empty set $\varnothing$ is often mistaken as representing nothingness. However, as a definable entity with formal properties and nonzero Kolmogorov complexity, it is not an instantiation of absolute nothingness but rather an empty object within a formal system \cite{Jech2003}.

\section{Philosophical and Technical Discussion}

\subsection{MIIP and Sufficient Reason}

MIIP reflects the ontological principle that existence entails information, echoing Leibniz’s Principle of Sufficient Reason \cite{Leibniz1714}. Contemporary work in digital physics \cite{Tegmark2008} aligns with this view, suggesting the universe’s fabric is informational at its core.

\subsection{Beyond Epistemology}

The argument addresses instantiability rather than mere epistemic limits. If $\mathcal{N}$ cannot be instantiated without contradiction, it is not simply unknown or indescribable but logically incoherent.

\subsection{Modal and Ontological Implications}

In modal logic terms,
\[
\neg \Diamond \mathcal{N},
\]
indicating that absolute nothingness is impossible. This underpins classical metaphysical intuitions that something must exist.

\section{Conclusion}

By synthesizing formal set theory, algorithmic information theory, and an original metaphysical principle, this paper establishes that absolute nothingness is logically inconsistent. Any instantiable system must contain minimal information; existence is fundamentally informational.

\begin{thebibliography}{99}

\bibitem{Chaitin1966}
Chaitin, G. J. (1966). On the length of programs for computing finite binary sequences. \textit{Journal of the ACM}, 13(4), 547--569.

\bibitem{Heidegger1962}
Heidegger, M. (1962). \textit{Being and Time} (J. Macquarrie \& E. Robinson, Trans.). Harper \& Row. (Original work published 1927)

\bibitem{Jech2003}
Jech, T. (2003). \textit{Set Theory}. Springer.

\bibitem{Kolmogorov1965}
Kolmogorov, A. N. (1965). Three approaches to the quantitative definition of information. \textit{Problems of Information Transmission}, 1(1), 1--7.

\bibitem{Leibniz1714}
Leibniz, G. W. (1714). \textit{Monadology}.

\bibitem{LiVitanyi2008}
Li, M., \& Vitányi, P. (2008). \textit{An Introduction to Kolmogorov Complexity and Its Applications} (3rd ed.). Springer.

\bibitem{Quine1948}
Quine, W. V. (1948). On what there is. \textit{Review of Metaphysics}, 2(5), 21--38.

\bibitem{Tegmark2008}
Tegmark, M. (2008). The mathematical universe. \textit{Foundations of Physics}, 38(2), 101--150.

\end{thebibliography}

\end{document}
